% =========================================================================
% SciPost LaTeX template
% Version 2024-07
%
% Submissions to SciPost Journals should make use of this template.
%
% INSTRUCTIONS: simply look for the `TODO:' tokens and adapt your file.
% ========================================================================

\documentclass{SciPost}

% Prevent all line breaks in inline equations.
\binoppenalty=10000
\relpenalty=10000

\hypersetup{
    colorlinks,
    linkcolor={red!50!black},
    citecolor={blue!50!black},
    urlcolor={blue!80!black}
}

\usepackage[bitstream-charter]{mathdesign}
\urlstyle{same}

% Fix \cal and \mathcal characters look (so it's not the same as \mathscr)
\DeclareSymbolFont{usualmathcal}{OMS}{cmsy}{m}{n}
\DeclareSymbolFontAlphabet{\mathcal}{usualmathcal}

\fancypagestyle{SPstyle}{
\fancyhf{}
\lhead{\colorbox{scipostblue}{\bf \color{white} ~SciPost Physics Codebases }}
\rhead{{\bf \color{scipostdeepblue} ~Submission }}
\renewcommand{\headrulewidth}{1pt}
\fancyfoot[C]{\textbf{\thepage}}
}


%\input mesmacros
   \newcommand{\dif}{\mathrm{d}}
   \newcommand{\bq}{\begin{eqnarray*}}
   \newcommand{\eq}{\end{eqnarray*}}
   \newcommand{\beq}{\begin{eqnarray}}
   \newcommand{\enq}{\end{eqnarray}}



\begin{document}

\pagestyle{SPstyle}

\begin{center}{\Large \textbf{\color{scipostdeepblue}{
%%%%%%%%%% TODO: Write your article's title here
PrimordialPy: a Python library for computing primordial power spectrum and PBHs abundances in single-field inflation
%%%%%%%%%% END TODO: TITLE
}}}\end{center}

\begin{center}\textbf{
%%%%%%%%%% TODO: AUTHORS
% Write the author list here. 
% Use (full) first name (+ middle name initials) + surname format.
% Separate subsequent authors by a comma, omit comma and use "and" for the last author.
% Mark the corresponding author(s) with a superscript symbol in this order
% \star, \dagger, \ddagger, \circ, \S, \P, \parallel, ...
Flavio Pineda\textsuperscript{1$\star$}
Luis O. Pimentel\textsuperscript{1$\dagger$}
%%%%%%%%%% END TODO: AUTHORS
}\end{center}

\begin{center}
%%%%%%%%%% TODO: AFFILIATIONS
% Write all affiliations here.
% Format: institute, city, country
{\bf 1} Departamento de Física, Universidad Aut\'onoma Metropolitana Iztapalapa P. O. Box 55-534, 09340 M\'exico, CDMX., M\'exico
%%%%%%%%%% END TODO: AFFILIATIONS
%%%%%%%%%% TODO: EMAIL
% Provide email address of corresponding author(s)
\\[\baselineskip]
$\star$ \href{mailto:email1}{\small fpineda@xanum.uam.mx}
$\dagger$ \href{mailto:email1}{\small lopr@xanum.uam.mx}

%%%%%%%%%% END TODO: EMAIL
\end{center}


\section*{\color{scipostdeepblue}{Abstract}}
\textbf{\boldmath{%
%%%%%%%%%% TODO: ABSTRACT
% Write your abstract here.
We present PrimordialPy, an open-source Python library designed to streamline the analysis of single-field inflationary models. Unlike existing tools often integrated into larger Boltzmann solvers, PrimordialPy provides a lightweight, modular, and object-oriented framework focused specifically on the inflationary epoch. The library numerically solves the background inflaton evolution and the Mukhanov-Sasaki equations for linear perturbations without relying on slow-roll approximations, ensuring accuracy for models with non-trivial kinetic terms or potential features. A key distinction of PrimordialPy is its dedicated module for computing the abundance of Primordial Black Holes (PBHs) directly from the primordial power spectrum, facilitating constraints on dark matter scenarios from inflation. We detail the numerical implementation, validate the code against standard benchmarks, and demonstrate its usage through examples. The source code is fully documented, unit-tested, and available at {\href{https://github.com/fpineda94/primordialpy.git}{\texttt{GitHub}}}, allowing researchers to easily extend the framework to new phenomenological models.
%%%%%%%%%% END TODO: ABSTRACT
}}

\vspace{\baselineskip}

%%%%%%%%%% BLOCK: Copyright information
% This block will be filled during the proof stage, and finilized just before publication.
% It exists here only as a placeholder, and should not be modified by authors.
\noindent\textcolor{white!90!black}{%
\fbox{\parbox{0.975\linewidth}{%
\textcolor{white!40!black}{\begin{tabular}{lr}%
  \begin{minipage}{0.6\textwidth}%
    {\small Copyright attribution to authors. \newline
    This work is a submission to SciPost Physics Codebases. \newline
    License information to appear upon publication. \newline
    Publication information to appear upon publication.}
  \end{minipage} & \begin{minipage}{0.4\textwidth}
    {\small Received Date \newline Accepted Date \newline Published Date}%
  \end{minipage}
\end{tabular}}
}}
}
%%%%%%%%%% BLOCK: Copyright information


%%%%%%%%%% TODO: LINENO
% For convenience during refereeing we turn on line numbers:
\linenumbers
% You should run LaTeX twice in order for the line numbers to appear.
%%%%%%%%%% END TODO: LINENO

%%%%%%%%%% TODO: TOC 
% Guideline: if your paper is longer that 6 pages, include a TOC
% To remove the TOC, simply cut the following block
\vspace{10pt}
\noindent\rule{\textwidth}{1pt}
\tableofcontents
\noindent\rule{\textwidth}{1pt}
\vspace{10pt}
%%%%%%%%%% END TODO: TOC


%%%%%%%%% TODO: CONTENTS 
% Write your article contents here, starting from first \section.
% An example structure is given below.

\section{Introduction}
\label{sec:intro}
% TODO: write your article here.

La inflación cósmica se ha convertido en la piedra angular y el esquema teórico lider para explorar la física del universo temprano, proporcionando una teoría solida sobre el origen y evolución de las perturbaciones primordiales y de la formación de estructra a gran escala. De acuerdo con esta teoría, el universo sufrió una expansión acelerada en una época muy temprana,



\section{Physics of single-field inflation}


\section{Primordial black holes (PBHs) formation}
\label{sec:another}

\subsection{Abundance of PBHs}






\section{Structure of the library}


\section{Examples}
  
  \subsection{Starobinsky inflation}
  \subsection{$\alpha$-attractors}
  \subsection{KKLT inflation with a bump}



\bibliography{biblio.bib}

%%%%%%%%%% END TODO: BIBLIOGRAPHY


\end{document}
